\begin{table}[htbp!]
\centering
\caption{Kruskal-Wallis H-Test Results for Feature Differentiation Across Clusters (RQ1)}
\label{app:rq1_cluster_validation_table}
\small
\setlength{\tabcolsep}{3pt}
\renewcommand{\arraystretch}{1}
\begin{threeparttable}
\begin{tabular*}{\textwidth}{l@{\extracolsep{\fill}} S[table-format=5.3] S[table-format=1.3,table-comparator=true]}
\toprule
\sbf{Feature} & {\sbf{H-statistic}} & {\sbf{p-value}} \\
\dmidrule
Perf.  & 232.613 & < .001 \\
Access.  & 920.863 & < .001 \\
Best Prac.  & 171.951 & < .001 \\
SEO  & 1756.283 & < .001 \\
Q-Title  & 14464.000 & < .001 \\
Q-H1  & 14464.000 & < .001 \\
ExQ-Title  & 12798.195 & < .001 \\
ExQ-H1  & 9102.510 & < .001 \\
Q/B Density  & 4388.283 & < .001 \\
Sim. Title  & 7561.791 & < .001 \\
Sim. Content  & 4641.727 & < .001 \\
Word Count  & 2015.283 & < .001 \\
\bottomrule
\end{tabular*}
\begin{tablenotes}[flushleft]
\scriptsize
\item All p-values reported as $< .001$ were originally $0.0$ in the source data.
\item Metric abbreviations are defined in Table \ref{tab:dataset_columns_types}.
\end{tablenotes}
\end{threeparttable}
\end{table}
